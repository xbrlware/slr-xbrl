% $Header$

\documentclass[t,14pt,mathserif]{beamer}

% This file is a solution template for:

% - Talk at a conference/colloquium.
% - Talk length is about 20min.
% - Style is ornate.



% Copyright 2004 by Till Tantau <tantau@users.sourceforge.net>.
%
% In principle, this file can be redistributed and/or modified under
% the terms of the GNU Public License, version 2.
%
% However, this file is supposed to be a template to be modified
% for your own needs. For this reason, if you use this file as a
% template and not specifically distribute it as part of a another
% package/program, I grant the extra permission to freely copy and
% modify this file as you see fit and even to delete this copyright
% notice.


\input{style.tex}

\usepackage[brazil]{babel}
%\usepackage[english]{babel}

\usepackage{graphicx}   %Package para figuras
% or whatever

\usepackage[utf8]{inputenc}
% or whatever

\usepackage{times}
\usepackage[T1]{fontenc}
\usepackage{tabularx}
\usepackage{multirow}
\usepackage{adjustbox}
\usepackage{array}
%\usepackage[cmex10]{amsmath}
% Or whatever. Note that the encoding and the font should match. If T1
% does not look nice, try deleting the line with the fontenc.

\newcommand{\semitransp}[2][35]{\color{fg!#1}#2}

\title[] % (optional, use only with long paper titles)
{Ferramentas para Business Report \\
com Suporte à Linguagem XBRL - \\
Revisão Sistemática da Literatura}

\subtitle
{Vagner Clementino}

%\author[] % (optional, use only with lots of authors)
%{Vagner Clementino~\inst{1}} %\and S.~Another\inst{2}}
% - Give the names in the same order as the appear in the paper.
% - Use the \inst{?} command only if the authors have different
%   affiliation.

\institute[] % (optional, but mostly needed)
{
%  \inst{1}%
  Departamento de Ciência da Computação\\
  Universidade Federal de Minas Gerais (UFMG)\\
  Empirical Software Engineering - 2015\\
  %\and
  %\inst{2}%
  %Department of Theoretical Philosophy\\
  %University of Elsewhere
  }
% - Use the \inst command only if there are several affiliations.
% - Keep it simple, no one is interested in your street address.

\date[2015/12/16] %o(optional, should be abbreviation of conference name)
%{Software Quality and Measurement 2015-1 \\Prof. Eduardo Figueiredo}
% - Either use conference name or its abbreviation.
% - Not really informative to the audience, more for people (including
%   yourself) who are reading the slides online

\subject{Software Engineer}
% This is only inserted into the PDF information catalog. Can be left
% out.



% If you have a file called "university-logo-filename.xxx", where xxx
% is a graphic format that can be processed by latex or pdflatex,
% resp., then you can add a logo as follows:

% \pgfdeclareimage[height=0.5cm]{university-logo}{university-logo-filename}
% \logo{\pgfuseimage{university-logo}}



% Delete this, if you do not want the table of contents to pop up at
% the beginning of each subsection:
\AtBeginSubsection[]
{
  \begin{frame}<beamer>{Outline}[allowframebreaks]
    \tableofcontents[currentsection,currentsubsection]
  \end{frame}
}


% If you wish to uncover everything in a step-wise fashion, uncomment
% the following command:

%\beamerdefaultoverlayspecification{<+->}

\expandafter\def\expandafter\insertshorttitle\expandafter{%
  \insertshorttitle\hfill%
  \insertframenumber\,/\,\inserttotalframenumber}

\setbeamertemplate{caption}[numbered]
\setbeamertemplate{bibliography item}{\insertbiblabel}
\begin{document}

\begin{frame}
  \titlepage
\end{frame}

\begin{frame}{Outline}
  \tableofcontents
  % You might wish to add the option [pausesections]
\end{frame}


% Structuring a talk is a difficult task and the following structure
% may not be suitable. Here are some rules that apply for this
% solution:

% - Exactly two or three sections (other than the summary).
% - At *most* three subsections per section.
% - Talk about 30s to 2min per frame. So there should be between about
%   15 and 30 frames, all told.

% - A conference audience is likely to know very little of what you
%   are going to talk about. So *simplify*!
% - In a 20min talk, getting the main ideas across is hard
%   enough. Leave out details, even if it means being less precise than
%   you think necessary.
% - If you omit details that are vital to the proof/implementation,
%   just say so once. Everybody will be happy with that.
\section{Introdução}
\section{Background}
\begin{frame}{eXtensible Business Report Language - XBRL}

    \begin{itemize}
      \item \alert{XBRL} é uma linguagem para divulgação e intercâmbio de
        informações financeiras baseada em XML \cite{xbrl_conceitos_aplicacoes}.
    \end{itemize}



\end{frame}

\section{Justificativa}
\section{Objetivo}
\section{Metodologia}
\subsection{Protocolo da Revisão}
\subsection{Critérios de Inclusão e Exclusão}
\subsection{Seleção dos Estudos Primários}
\subsection{Decisão de Inclusão e Exclusão}
\subsection{Análise da Qualidade dos Estudos}
\subsection{Extração dos Dados}
\subsection{Sintetização dos Dados}

\section{Resultados}
\section{Ameaças à Validade}
\section{Conclusão}


\begin{frame}[allowframebreaks]
   \frametitle{Referências}
   \bibliographystyle{IEEEtranS}
   \bibliography{IEEEfull,bibliografia}
\end{frame}

\end{document}
