% $Header$

\documentclass[t,14pt,mathserif]{beamer}

% This file is a solution template for:

% - Talk at a conference/colloquium.
% - Talk length is about 20min.
% - Style is ornate.



% Copyright 2004 by Till Tantau <tantau@users.sourceforge.net>.
%
% In principle, this file can be redistributed and/or modified under
% the terms of the GNU Public License, version 2.
%
% However, this file is supposed to be a template to be modified
% for your own needs. For this reason, if you use this file as a
% template and not specifically distribute it as part of a another
% package/program, I grant the extra permission to freely copy and
% modify this file as you see fit and even to delete this copyright
% notice.


\input{style.tex}

\usepackage[brazil]{babel}
%\usepackage[english]{babel}

\usepackage{graphicx}   %Package para figuras
% or whatever

\usepackage[utf8]{inputenc}
% or whatever

\usepackage{times}
\usepackage[T1]{fontenc}
\usepackage{tabularx}
\usepackage{multirow}
\usepackage{adjustbox}
\usepackage{array}
%\usepackage[cmex10]{amsmath}
% Or whatever. Note that the encoding and the font should match. If T1
% does not look nice, try deleting the line with the fontenc.

\newcommand{\semitransp}[2][35]{\color{fg!#1}#2}

\title[] % (optional, use only with long paper titles)
{Ferramentas para Business Report \\
com Suporte à Linguagem XBRL - \\
Revisão Sistemática da Literatura}

\subtitle
{Vagner Clementino}

%\author[] % (optional, use only with lots of authors)
%{Vagner Clementino~\inst{1}} %\and S.~Another\inst{2}}
% - Give the names in the same order as the appear in the paper.
% - Use the \inst{?} command only if the authors have different
%   affiliation.

\institute[] % (optional, but mostly needed)
{
%  \inst{1}%
  Departamento de Ciência da Computação\\
  Universidade Federal de Minas Gerais (UFMG)\\
  Empirical Software Engineering - 2015\\
  %\and
  %\inst{2}%
  %Department of Theoretical Philosophy\\
  %University of Elsewhere
  }
% - Use the \inst command only if there are several affiliations.
% - Keep it simple, no one is interested in your street address.

\date[2015/12/16] %o(optional, should be abbreviation of conference name)
%{Software Quality and Measurement 2015-1 \\Prof. Eduardo Figueiredo}
% - Either use conference name or its abbreviation.
% - Not really informative to the audience, more for people (including
%   yourself) who are reading the slides online

\subject{Software Engineer}
% This is only inserted into the PDF information catalog. Can be left
% out.



% If you have a file called "university-logo-filename.xxx", where xxx
% is a graphic format that can be processed by latex or pdflatex,
% resp., then you can add a logo as follows:

% \pgfdeclareimage[height=0.5cm]{university-logo}{university-logo-filename}
% \logo{\pgfuseimage{university-logo}}



% Delete this, if you do not want the table of contents to pop up at
% the beginning of each subsection:
\AtBeginSubsection[]
{
  \begin{frame}<beamer>{Outline}[allowframebreaks]
    \tableofcontents[currentsection,currentsubsection]
  \end{frame}
}


% If you wish to uncover everything in a step-wise fashion, uncomment
% the following command:

%\beamerdefaultoverlayspecification{<+->}

\expandafter\def\expandafter\insertshorttitle\expandafter{%
  \insertshorttitle\hfill%
  \insertframenumber\,/\,\inserttotalframenumber}

\setbeamertemplate{caption}[numbered]
\setbeamertemplate{bibliography item}{\insertbiblabel}
\begin{document}

\begin{frame}
  \titlepage
\end{frame}

\begin{frame}{Outline}
  \tableofcontents
  % You might wish to add the option [pausesections]
\end{frame}


% Structuring a talk is a difficult task and the following structure
% may not be suitable. Here are some rules that apply for this
% solution:

% - Exactly two or three sections (other than the summary).
% - At *most* three subsections per section.
% - Talk about 30s to 2min per frame. So there should be between about
%   15 and 30 frames, all told.

% - A conference audience is likely to know very little of what you
%   are going to talk about. So *simplify*!
% - In a 20min talk, getting the main ideas across is hard
%   enough. Leave out details, even if it means being less precise than
%   you think necessary.
% - If you omit details that are vital to the proof/implementation,
%   just say so once. Everybody will be happy with that.
\section{Introdução}
\section{Background}
\begin{frame}{eXtensible Business Report Language - XBRL}

    \begin{itemize}
      \item \alert{XBRL} é uma linguagem para divulgação e intercâmbio de
        informações financeiras baseada em XML \cite{xbrl_conceitos_aplicacoes}.
    \end{itemize}



\end{frame}

\section{Justificativa}
\section{Objetivo}
\section{Metodologia}
%%%%%%%%%%%%%%%%%%%%%%%%%%%%%%%%%%%%%%%%%%%%%%%%%%%%%%%%
\begin{frame}{Protocolo da Revisão}
    \begin{itemize}
      \item Definição de um Protocolo com diretrizes para SLR
        \cite{kitchenham2009systematic}
      \item Foram propostas Questões de Pesquisa a serem respondidas pela SLR
  \end{itemize}
\end{frame}
%%%%%%%%%%%%%%%%%%%%%%%%%%%%%%%%%%%%%%%%%%%%%%%%%%%%%%%%%
%%%%%%%%%%%%%%%%%%%%%%%%%%%%%%%%%%%%%%%%%%%%%%%%%%%%%%%%
\begin{frame}{Questões de Pesquisa}
    \begin{itemize}
     \item \textbf{$Q1$}: Quais são as ferramentas para Relatórios de Negócio que
    suportam a XBRL?
  \item \textbf{$Q2$}: Quais são as funcionalidades comuns as ferramentas
    que possibilitem a comparação entre elas?
    \end{itemize}
\end{frame}
%%%%%%%%%%%%%%%%%%%%%%%%%%%%%%%%%%%%%%%%%%%%%%%%%%%%%%%%%
%%%%%%%%%%%%%%%%%%%%%%%%%%%%%%%%%%%%%%%%%%%%%%%%%%%%%%%%
\begin{frame}{Questões de Pesquisa}
    \begin{itemize}
        \item \textbf{$Q3$}: Existem casos reais de utilização da ferramenta
    (Estudos de Casos, Whitepapers e etc)?
      \item \textbf{$Q4$}: Qual setor da economia (governos, medicina, setor financeiro) a ferramenta possui histórico de utilização?
    \end{itemize}
\end{frame}
%%%%%%%%%%%%%%%%%%%%%%%%%%%%%%%%%%%%%%%%%%%%%%%%%%%%%%%%%
%%%%%%%%%%%%%%%%%%%%%%%%%%%%%%%%%%%%%%%%%%%%%%%%%%%%%%%%
\begin{frame}{Critérios de Inclusão e Exclusão}
    \begin{itemize}
      \item Critérios de \textit{inclusão}
              \begin{itemize}

                   \item Tem ser publicado a partir de 2008.
                   \item Estar escrito em língua inglesa.
                   \item Artigos de Conferência, journals e Whitepapers
                   \item Dissertações ou Teses apenas se a ferramenta proposta tenha sido
  implementada e testada.
              \end{itemize}
    \end{itemize}
\end{frame}
%%%%%%%%%%%%%%%%%%%%%%%%%%%%%%%%%%%%%%%%%%%%%%%%%%%%%%%%%
%%%%%%%%%%%%%%%%%%%%%%%%%%%%%%%%%%%%%%%%%%%%%%%%%%%%%%%%
\begin{frame}{Critérios de Inclusão e Exclusão}
    \begin{itemize}
      \item Critérios de \textit{exclusão}
               \begin{itemize}
                       \item Trabalhos escritos em outra língua que não a inglesa
                       \item Documentos duplicados.
                       \item Livros
                       \item Dissertações ou Teses apenas se a ferramenta que
                         não há a implementação da ferramenta.
                       \item Literaturas escritas antes do ano de 2008
              \end{itemize}
      \end{itemize}
\end{frame}
%%%%%%%%%%%%%%%%%%%%%%%%%%%%%%%%%%%%%%%%%%%%%%%%%%%%%%%%%

%%%%%%%%%%%%%%%%%%%%%%%%%%%%%%%%%%%%%%%%%%%%%%%%%%%%%%%%
\begin{frame}{Seleção dos Estudos Primários}
\begin{itemize}
    \item Base de dados conforme Brereton et. al. \cite{Brereton2007571} com
      pequenas alterações
\end{itemize}

\begin{table}[ht]
\centering
\resizebox{6.9cm}{!}{%
\begin{tabular}{|c|l|c|c|}
\hline
\textbf{\#} & \multicolumn{1}{c|}{\textbf{Base de Dados}} & \textbf{Total} & \textbf{Percentual} \\ \hline
1           & IEEE Xplore                                 & 3              & 0,74\%      \\ \hline
2           & ScienceDirect                               & 100            & 24,57\%     \\ \hline
3           & Springer Link                               & 6              & 1,47\%      \\ \hline
4           & ACM Digital Library                         & 97             & 23,83\%     \\ \hline
5           & Web of Science                              & 9              & 2,21\%      \\ \hline
6           & CiteSeer                                    & 45             & 11,06\%     \\ \hline
7           & Wiley Online Library                        & 54             & 13,27\%     \\ \hline
8           & Scopus Elsevier                             & 8              & 1,97\%      \\ \hline
9           & EL Compendex                                & 9              & 2,21\%      \\ \hline
10          & Google scholar                              & 30             & 7,37\%      \\ \hline
11          & XBRL Consortium                             & 46             & 11,30\%     \\ \hline
\multicolumn{2}{|c|}{Total}                               & 407            & 100,00\%    \\ \hline
\end{tabular}
}
\caption{Base de dados e número de artigos}
\label{tab:base-dados}
\end{table}
\end{frame}
%%%%%%%%%%%%%%%%%%%%%%%%%%%%%%%%%%%%%%%%%%%%%%%%%%%%%%%%%
\begin{frame}{Sentença de Busca}
\begin{itemize}
    \item ``XBRL \textbf{AND} Business Report \textbf{AND} tool''
    \item Utilização de Tabela de Sinônimos
    \item

\end{itemize}
\begin{table}[ht]
\centering
\resizebox{\textwidth}{!}{%
\begin{tabular}{|c|l|}
\hline
\multicolumn{2}{|c|}{\textbf{DICIONÁRIO DE SINÔNIMOS}} \\ \hline
\textbf{Termo Original} & \multicolumn{1}{c|}{\textbf{Sinônimo}} \\ \hline
XBRL & XML OR XHTML \\ \hline
tool & sofwtare OR  application OR product OR project OR development \\ \hline
Business Report & Finantial Report OR Data Extraction \\ \hline
\end{tabular}
}
\caption{Dicionário de Sinônimos}
\label{tab:dicionario}
\end{table}

\end{frame}
%%%%%%%%%%%%%%%%%%%%%%%%%%%%%%%%%%%%%%%%%%%%%%%%%%%%%%%%%

%%%%%%%%%%%%%%%%%%%%%%%%%%%%%%%%%%%%%%%%%%%%%%%%%%%%%%%%
\begin{frame}{Decisão de Inclusão e Exclusão}
    \begin{itemize}
      \item
    \end{itemize}
\end{frame}
%%%%%%%%%%%%%%%%%%%%%%%%%%%%%%%%%%%%%%%%%%%%%%%%%%%%%%%%%

%%%%%%%%%%%%%%%%%%%%%%%%%%%%%%%%%%%%%%%%%%%%%%%%%%%%%%%%
\begin{frame}{Análise da Qualidade dos Estudos}
    \begin{itemize}
      \item
    \end{itemize}
\end{frame}
%%%%%%%%%%%%%%%%%%%%%%%%%%%%%%%%%%%%%%%%%%%%%%%%%%%%%%%%%

%%%%%%%%%%%%%%%%%%%%%%%%%%%%%%%%%%%%%%%%%%%%%%%%%%%%%%%%
\begin{frame}{Extração dos Dados}
    \begin{itemize}
      \item
    \end{itemize}
\end{frame}
%%%%%%%%%%%%%%%%%%%%%%%%%%%%%%%%%%%%%%%%%%%%%%%%%%%%%%%%%

%%%%%%%%%%%%%%%%%%%%%%%%%%%%%%%%%%%%%%%%%%%%%%%%%%%%%%%%
\begin{frame}{Sintetização dos Dados}
    \begin{itemize}
      \item
    \end{itemize}
\end{frame}
%%%%%%%%%%%%%%%%%%%%%%%%%%%%%%%%%%%%%%%%%%%%%%%%%%%%%%%%%

%%%%%%%%%%%%%%%%%%%%%%%%%%%%%%%%%%%%%%%%%%%%%%%%%%%%%%%%
\begin{frame}{}
    \begin{itemize}
      \item
    \end{itemize}
\end{frame}
%%%%%%%%%%%%%%%%%%%%%%%%%%%%%%%%%%%%%%%%%%%%%%%%%%%%%%%%%

\begin{frame}[allowframebreaks]
   \frametitle{Referências}
   \bibliographystyle{IEEEtranS}
   \bibliography{IEEEfull,bibliografia}
\end{frame}

\end{document}
